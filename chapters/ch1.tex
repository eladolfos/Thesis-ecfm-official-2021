\chapter{OBJETOS COMPACTOS}

En astrofis\'ica, se le denomina objetos compactos a los remanentes de estrellas, estos objetos ``nacen'' cuando las estrellas ``mueren'', estos pueden ser enanas blancas, estrellas de neutrones y agujeros negros.  En el caso de los agujeros negros supermasivos,  se considera que estos habitan el centro de cada galaxia. Estos objetos por ser los remanentes de la estrellas, pueden encontrarse por toda la galaxia \citep{1983-shapiro,Camenzind2007}.\\

Los primeros dos objetos surgen cuando una estrella termina su combustible nuclear y entonces ya no puede soportar la fuerza gravitacional que la hace colapsar con sigo misma, en este colapso se genera una presi\'on t\'ermica que evita el colapso, que en el caso de las Enanas Blancas la presi\'on es debida a electrones, y para el caso de las Estrellas de Neutrones es debida a neutrones.  Por otro lado los agujeros negros surgen cuando la fuerza gravitacional es tan grande que no existe nada que evite el colapso sobre si misma generando as\'i una singularidad en el espacio tiempo.

\section{Enanas Blancas}

\section{Estrellas de Neutrones}
Las Estrellas de Neutrones (NS por sus siglas en ingles) constituyen uno de los objetos mas intrigantes en el universo, su nombre se debe a que se encuentran compuestas en su mayoría por neutrones

Los primeros c\'alculos teóricos fueren hechor por \citep{1939Oppenheimer-Neutron-Cores}, inspirados por el trabajo hecho con las WDs ellos modelaron la materia de la estrella como un gas degenerado de neutrones, cuya presi\'on de degeneraci\'on es la responsable de que la estrella no colapse sobre si misma. Ellos calcularon estrellas con masa máxima de $\sim 0.7~\Ms$ (es decir por debajo del limite de Chandrsekhar), densidades promedio arriba $\num{6e15}\g~\cm^{-3}$ y radios de $\sim 10 ~\km$  \citep{2000Heiselber-Recent-Progress}.  Sin embargo hoy en d\'ia se sabe que las estrellas de neutrones si pueden exceder el limite de Chandrasekhar por lo que sus masas pueden encontrarse entre $1.0$-$2.14\Ms$ con densidades que incrementan con la profundidad, las cuales varían desde $\sim 10^6~\g~\cm^{-3}$ hasta $\sim \num{8e14}~\g~\cm^{-3}$. Las estrellas de neutrones poseen radios de $\sim 12~\km$ \citep{Camenzind2007,2011-rawls-mass-1Msolar,neutron-star-masss,2019-fonseca-mass214}. Uno de los mayores problemas  es el estado de los nucleones que la componen, ya que éstos pueden estar unidos en núcleos o estar libres en estados continuos \citep{Camenzind2007}. 


\section{P\'ulsar}

Los púlsares son estrellas de neutrones que giran y emiten radiación de forma periódica, poseen intensos campos magnéticos y gravitacionales  y periodos de rotación que van desde $\sim 10^{-3}$ hasta  $\sim25$ segundos\footnote{Valores obtenidos de \url{https://www.atnf.csiro.au/people/pulsar/psrcat/} el 30/03/2020}. 

Las estrellas progenitoras \citep{2017PhDT-Golam}

\subsection{Pulsar de Milisegundo}
En particular los púlsares de milisegundo  (MSPs por sus siglas en Ingles) son una subclase de púlsares que emiten en radio, poseen periodos cortos menores a 30 ms. Los MSPs pueden encontrarse en cúmulos globulares, por ejemplo PSR B1821−24A en el cúmulo M28, los MSPs ademas emiten pulsos relativamente fuertes en gamma ($\gamma$-rays) la primera evidencia de emisión en rayos gamma fue observada del pulsar PSR J0218+4232 y reportada por \citet{2000kuipler-fst-gamma-ray-pulsar}, este era un p\'ulsar binario muy conocido con un periodo de 2.3 ms \citep{2017Manchester-MSP-Evolution-aplications}. Estos MSPs que emiten gamma poseen características inusuales en comparación con otras fuentes que emiten en gamma, estas son según \citet{2017Manchester-MSP-Evolution-aplications}:

\begin{itemize}
	\item Son emisores estables durante largos intervalos y tienen espectros característicos que siguen una de ley de potencia con un corte exponencial a unos pocos GeV
	\item Estas pulsaciones de rayos γ se han detectado posteriormente al unir los datos de rayos $\gamma$ con el período preciso de efemérides de las observaciones de radio.
\end{itemize} 
otro aspecto singular de los MSPs es que, muchos de estos se han encontrado con periodos binarios cortos ($P_b\lesssim 1$ día) y compa\~neras con masas peque\~nas ($M_c\lesssim 0.3 \Ms$) y exhiben eclipses de radio debido al gas circundante de la compa\~nera, formando sistemas del tipo Black Widows (BWs) o  Redbacks (RB).

\subsection{P\'ulsares Reciclados}
Los p\'ulsar de milisegundo (MSP) son una clase de estrellas de neutrones viejas, que son caracterizadas por  periodos ($P$) de rotación cortos y estables típicamente $P<30$ ms \citep{Zharikov2019}.   

Los p\'ulsares reciclado (``Recycled Pulsars'') MSPs son el resultado del reciclaje de pulsares viejos, estos se encuentran en sistemas binarios, cuando la compañera transfiere masa también transfiere momentun angular desde la órbita hasta la estrella de neutrones, girándola y reactivando el proceso de emisión de púlsar \citep[e.g.,]{2017Manchester-MSP-Evolution-aplications}. 

La única forma conocida para que se forme un MSP es con la ayuda de otra estrella, sin embargo observamos MSP sin compañeras binarias. Estos ``Recycled Pulsars'' debieron haberse formado en sistemas binarios, pero ahora se encuentran solos \citep{Crowter_2018}.  

\section{RedBacks}


\section{Black Widows Pulsars}

Un Black Widow pulsar  son sistemas binarios eclipsantes, este termino fue acu\~nado por \citet{1988Eichler-On-Black-widow} \\

En los sistemas denominados Black Widows (BW), un pulsar de milisegundo es acompañado por una estrella degenerada de masa pequeña y poco densa, que se encuentra muy cercana, la masa es alrededor de 0.01$\Ms$ y que se esta hinchando y además es fuertemente irradiado por el pulsar, esto  conduce a que existan emisiones lo suficientemente fuertes como para eclipsar el pulsar en fracciones de su órbita \citep{van2011,Crowter_2018}. Los BWs se caracterizan por la destrucción de la compañera del pulsar. Los pulsar emiten fuertes cantidades de partículas relativistas cargadas la cuales al incidir en la compañera terminan destruyéndola \citep{Crowter_2018}.\\  

El primer descubrimiento de un sistema de este tipo fue hecho por \citet{fruchter1988} con el pulsar PSR B1957+20 observado en radio a 430 MHz en el Observatorio de Arecibo\footnote{\url{https://www.naic.edu/ao/}}: es un pulsar de milisegundo con un periodo de 1.6 ms y un periodo binario de 9.17 h.  Se observó que la señal del púlsar se eclipsó durante aproximadamente 50 minutos en cada órbita, y durante los pocos minutos que precedieron a un eclipse y durante al menos 20 después, la señal se retrasó. La masa minima de la compañera fue de 0.022 $\Ms$ Este pulsar emite haces de radiación que inciden sobre una estrella compañera (enana cafe ligera).



\section{Un sistema binario eclipsante}

Los sistemas binarios son sistemas formados por dos estrellas que giran alrededor del centro de masa. Los sistema eclipsantes 